\documentclass[12pt,,letterpaper,twoside]{report}
\usepackage[]{graphicx}
\usepackage[]{color}

%% maxwidth is the original width if it is less than linewidth
%% otherwise use linewidth (to make sure the graphics do not exceed the margin)
\makeatletter
\def\maxwidth{ %
  \ifdim\Gin@nat@width>\linewidth
    \linewidth
  \else
    \Gin@nat@width
  \fi
}
\makeatother

\definecolor{fgcolor}{rgb}{0.345, 0.345, 0.345}
\newcommand{\hlnum}[1]{\textcolor[rgb]{0.686,0.059,0.569}{#1}}%
\newcommand{\hlstr}[1]{\textcolor[rgb]{0.192,0.494,0.8}{#1}}%
\newcommand{\hlcom}[1]{\textcolor[rgb]{0.678,0.584,0.686}{\textit{#1}}}%
\newcommand{\hlopt}[1]{\textcolor[rgb]{0,0,0}{#1}}%
\newcommand{\hlstd}[1]{\textcolor[rgb]{0.345,0.345,0.345}{#1}}%
\newcommand{\hlkwa}[1]{\textcolor[rgb]{0.161,0.373,0.58}{\textbf{#1}}}%
\newcommand{\hlkwb}[1]{\textcolor[rgb]{0.69,0.353,0.396}{#1}}%
\newcommand{\hlkwc}[1]{\textcolor[rgb]{0.333,0.667,0.333}{#1}}%
\newcommand{\hlkwd}[1]{\textcolor[rgb]{0.737,0.353,0.396}{\textbf{#1}}}%
\let\hlipl\hlkwb

\usepackage{framed}
\makeatletter
\newenvironment{kframe}{%
 \def\at@end@of@kframe{}%
 \ifinner\ifhmode%
  \def\at@end@of@kframe{\end{minipage}}%
  \begin{minipage}{\columnwidth}%
 \fi\fi%
 \def\FrameCommand##1{\hskip\@totalleftmargin \hskip-\fboxsep
 \colorbox{shadecolor}{##1}\hskip-\fboxsep
     % There is no \\@totalrightmargin, so:
     \hskip-\linewidth \hskip-\@totalleftmargin \hskip\columnwidth}%
 \MakeFramed {\advance\hsize-\width
   \@totalleftmargin\z@ \linewidth\hsize
   \@setminipage}}%
 {\par\unskip\endMakeFramed%
 \at@end@of@kframe}
\makeatother

\definecolor{shadecolor}{rgb}{.97, .97, .97}
\definecolor{messagecolor}{rgb}{0, 0, 0}
\definecolor{warningcolor}{rgb}{1, 0, 1}
\definecolor{errorcolor}{rgb}{1, 0, 0}
\newenvironment{knitrout}{}{} % an empty environment to be redefined in TeX

\usepackage{alltt}

% Layout
\usepackage{geometry}
\usepackage{setspace}
\usepackage{titlesec}
\usepackage[subfigure]{tocloft}
\usepackage{mdframed}
\usepackage{titletoc}

% Citation style
\usepackage{natbib}
\usepackage{apalike}

% include citations inline
\usepackage{bibentry}
\nobibliography*

% Figures
\usepackage{subfigure}
\usepackage{epsfig}
\usepackage{booktabs}
\usepackage{multicol}
\usepackage{listings}

% Math
\usepackage{amsthm}
\usepackage{amsmath}
\usepackage{amssymb}

% Typography
\usepackage{times}
\usepackage{microtype}
\usepackage{textcomp}

% Macro support
\usepackage{xspace}

% PDF links
\usepackage[hidelinks]{hyperref} % backref=page

% Some packages to help with tables/figures
\usepackage[section]{placeins}
\usepackage{morefloats}
\usepackage{amssymb}
\usepackage{setspace}
\usepackage{rotating}
\usepackage{graphicx}
\usepackage{booktabs} 
\usepackage{stmaryrd}
\usepackage{times}
\usepackage{bm}
\usepackage{indentfirst}

% Use this command to start each chapter
% The format allows chapters in the table 
% of contents to meet grad school requirements
% as of 2017

\newcommand{\mychapter}[2]{
    \setcounter{chapter}{#1}
    \setcounter{section}{0}
    \chapter*{#2}
    \addcontentsline{toc}{chapter}{#2}
}


\usepackage{appendix}

\input{DoctoralThesisTexTemplate/template/layout}

\input{DoctoralThesisTexTemplate/template/macros}

\IfFileExists{upquote.sty}{\usepackage{upquote}}{}

% You can put custom definitions and other LaTeX header information here
\DeclareMathOperator{\E}{{\rm I\kern-.3em E}}
\DeclareMathOperator{\I}{{\rm I}}
\DeclareMathOperator{\BR}{BR}
\DeclareMathOperator*{\argmax}{argmax}
\DeclareMathOperator*{\argmin}{argmin}
\newcommand\norm[1]{\left\lVert#1\right\rVert}

\begin{document}

%--------   BEGIN FRONTMATTER -------------------------------------------------%
% front matter pages use 2in top margin
\newgeometry{left=1.25in,top=2in,right=1.25in,bottom=1in,nohead}
\pagenumbering{roman}

%%%%%%%%%%%%%%%%%%%%%%%%%%%%%%%%% TITLEPAGE %%%%%%%%%%%%%%%%%%%%%%%%%%%%%%%%%%%%
\begin{titlepage}
\begin{center}

% 1. The title of the thesis/dissertation, centered 2? below the top of the page

\vspace{2in}
\begin{singlespace}
Experimental Designs for Pragmatic Trials and Precision Medicine
\end{singlespace}


% 2. Your name, centered 1? below the title.
\vspace{61pt} % 1 in = 72pt, 11pt for the line with text
John H. Sperger
\end{center}

\vspace{50pt}
\begin{singlespace}
\begin{center}
\noindent 
A dissertation submitted to the faculty of the University of North Carolina at Chapel Hill in partial fulfillment of the requirements for the degree of Doctor
of Philosophy in the Department of Biostatistics in the Gillings School
of Global Public Health.
\end{center}
\end{singlespace}


%4. On the lower half of the page, centered, the words ?Chapel Hill?
%and one line below that, the year in which your committee approves
%the completed thesis/dissertation.
\vspace{50pt}
\begin{center}
\begin{singlespace} 
Chapel Hill\\
2022
\end{singlespace}
\end{center}

%5. On the right-hand side of the page, ?Approved by,? followed by lines for the
%signatures of the adviser and four (two for thesis) readers. List
 

\vfill
\begin{flushright}
\begin{minipage}[t]{1.5in} 
Approved by:\\
%To be approved by: \\
 
 Dr.~Michael R. Kosorok \\  Dr.~Eric Laber \\ 
 Dr.~Anastasia Ivanova \\  Dr.~Lisa M. Lavange \\  Dr.~Angela Smith \\ 

\end{minipage}
\end{flushright}

\end{titlepage}

%%%%%%%%%%%%%%%%%%%%%%%%%%%%%%%%% COPYRIGHT %%%%%%%%%%%%%%%%%%%%%%%%%%%%%%%%%%%%

%2. Copyright Page (optional)
\newgeometry{left=1.25in,top=8.33in,right=1.25in,bottom=1in,nohead}

%If you wish to copyright your thesis, you must include a copyright page with the following information single-spaced and centered on the bottom half of the page:
%? Year 
%Full Name (exactly as it appears on the title page) 
%ALL RIGHTS RESERVED
%This page should immediately follow the title page, and should bear the lower case Roman numeral: ii.

\begin{center}
\begin{singlespace}
\copyright 2022 \\
John H. Sperger \\
ALL RIGHTS RESERVED
\end{singlespace}
\end{center}

\clearpage
\newgeometry{left=1.25in,top=2in,right=1.25in,bottom=1in,nohead}

% Normal pages from here on out; TOC title takes care of 2in requirement.
\restoregeometry

%%%%%%%%%%%%%%%%%%%%%%%%%%%%%%%%% ABSTRACT %%%%%%%%%%%%%%%%%%%%%%%%%%%%%%%%%%%%
%The word Abstract should be centered below the top of the page. 
%Skip one line, then center your name followed by the title of the 
%thesis/dissertation. Use as many lines as necessary. Centered below the 
%title include the phrase, in parentheses, ?(Under the direction of  
%_________)? and include the name(s) of the dissertation advisor(s).
%Skip one line and begin the content of the abstract. It should be 
%double-spaced and conform to margin guidelines. An abstract should not 
%exceed 150 words for a thesis and 350 words for a dissertation. The 
%latter is a requirement of both the Graduate School and UMI's 
%Dissertation Abstracts International.
%Because your dissertation abstract will be published, please prepare and 
%proofread it carefully. Print all symbols and foreign words clearly and 
%accurately to avoid errors or delays. Make sure that the title given at 
%the top of the abstract has the same wording as the title shown on your 
%title page. Avoid mathematical formulas, diagrams, and other 
%illustrative materials, and only offer the briefest possible description 
%of your thesis/dissertation and a concise summary of its conclusions. Do 
%not include lengthy explanations and opinions.
%The abstract should bear the lower case Roman number ii (if you did not 
%include a copyright page) or iii (if you include a copyright page).

\begin{center}
\vspace*{52pt}
{\normalsize \textbf{ABSTRACT}}
\vspace{11pt}

\begin{singlespace}
 John H. Sperger : Experimental Designs for Pragmatic Trials and
Precision Medicine \\
(Under the direction of   Dr.~Michael R. Kosorok and  Dr.~Eric Laber)
\end{singlespace}
\end{center}

Traditional clinical trial designs focus on demonstrating the efficacy
of a treatment compared to a placebo quantified in terms of the average
treatment effect (ATE). While these methods have been profoundly
successful, the guidance they provide is limited compared to the rich
complexity of information physicians consider when making their
treatment recommendations. In recent decades, two separate strands of
research have sought to build on these foundations while providing
evidence that can better inform clinical decision-making: 1) comparative
effectiveness studies that investigate the relative efficacy of multiple
treatments for the same condition in terms of the ATE, and 2) precision
medicine research that attempts to formalize the long-standing medical
practice of tailoring treatment based on an individual's unique
characteristics in a data-driven way. Equally allocating patients
between treatments is not an efficient way to investigate these new
research questions, and new experimental designs are needed to improve
the information gained from clinical trials.

We first tackle the question of how to design a trial when treatment effects vary depending
on a patient's characteristics and our goal is to estimate a dynamic
treatment regime (DTR) that will maximize the expected outcomes of the
general population. To this end, we propose a sequential design that
removes ineffective treatments at set intervals in the trial by
evaluating the change in the value function if that treatment weren't
available to assign in a DTR. This can be viewed as an extension of the
successive rejection algorithm from the response-adaptive randomization
(RAR) literature to the case where the expected response depends on
contextual patient information. We discuss a trial design problem that
motivated this design and conduct a simulation study to demonstrate the
effectiveness of this design compared to equal allocation. 

For further
projects, I propose to extend this design to the structured case where
one treatment can provide information about the expected response on
another treatment (e.g.~different doses of the same medication).
Eliminating ineffective treatments in the structured case will require
modifying the decision criteria to account for potential information
gain. I also propose a design for multi-level interventions that have an
intervention at the individual level that can be fully randomized and a
cluster-level intervention that is randomized using a stepped-wedge
design. This is motivated by the ongoing NC Works4Health trial to
evaluate the effectiveness of a multi-level intervention comprised of a
lifestyle coaching program at the individual level and supervisor
training at the employer level.

\clearpage

%%%%%%%%%%%%%%%%%%%%%%%%%%%%%%%%% DEDICATION %%%%%%%%%%%%%%%%%%%%%%%%%%%%%%%%%%%

%A dedication is an honorific statement from the author to a person or group to 
%whom the author commends the effort and product of the dissertation. Most 
%dedications are short statements of tribute beginning with ?To??. No heading is 
%required on the dedication page. The text of short dedications should be 
%centered between the left and right margins and 2? from the top of the page.
\begin{center}
\vspace*{52pt}

\singlespacing

I hope we passed the audition.

\end{center}

\pagebreak

%%%%%%%%%%%%%%%%%%%%%%%%%%%%%%%%% ACKNOWLEDGEMENTS %%%%%%%%%%%%%%%%%%%%%%%%%%%%%

%Acknowledgements are the author's statement of gratitude to and
%recognition of the people and institutions who helped the author's
%research and writing.
\begin{center}
\vspace*{52pt}
{\normalsize \textbf{ACKNOWLEDGEMENTS}}
\end{center}

Thanks grandma!

\clearpage

%%%%%%%%%%%%%%%%%%%%%%%%%%%%%%%%% TABLE OF CONTENTS %%%%%%%%%%%%%%%%%%%%%%%%%%%%

\renewcommand{\contentsname}{TABLE OF CONTENTS}
\renewcommand{\cfttoctitlefont}{\normalsize\bfseries}
\renewcommand{\cftaftertoctitle}{\hfill}
\renewcommand{\cftdotsep}{1.5}
\cftsetrmarg{1.0in}

\setlength{\cftbeforetoctitleskip}{61pt}
\setlength{\cftaftertoctitleskip}{28pt}



% format chapter entries like other entries
\renewcommand{\cftchapfont}{\normalfont}
\renewcommand{\cftchappagefont}{\normalfont}
\renewcommand{\cftchapleader}{\cftdotfill{\cftdotsep}}

\setlength{\cftbeforechapskip}{15pt}
\setlength{\cftbeforesecskip}{10pt}
\setlength{\cftbeforesubsecskip}{10pt}
\setlength{\cftbeforesubsubsecskip}{10pt}

\begin{singlespace}
\begin{center}
\tableofcontents
\end{center}
\end{singlespace}

\clearpage

%%%%%%%%%%%%%%%%%%%%%%%%%%%%%%%%% LIST OF TABLES %%%%%%%%%%%%%%%%%%%%%%%%%%%%%%%
\renewcommand{\listtablename}{LIST OF TABLES}
\phantomsection
\addcontentsline{toc}{chapter}{LIST OF TABLES}

\setlength{\cftbeforelottitleskip}{-11pt}
\setlength{\cftafterlottitleskip}{22pt}
\renewcommand{\cftlottitlefont}{\hfill\normalsize\bfseries}
\renewcommand{\cftafterlottitle}{\hfill}

\setlength{\cftbeforetabskip}{10pt}

\begin{singlespace}
\listoftables
\end{singlespace}

\clearpage

%%%%%%%%%%%%%%%%%%%%%%%%%%%%%%%%% LIST OF FIGURES %%%%%%%%%%%%%%%%%%%%%%%%%%%%%%
\renewcommand{\listfigurename}{LIST OF FIGURES}
\phantomsection
\addcontentsline{toc}{chapter}{LIST OF FIGURES}

\setlength{\cftbeforeloftitleskip}{-11pt} %11
\setlength{\cftafterloftitleskip}{22pt} %22
\renewcommand{\cftloftitlefont}{\hfill\normalsize\bfseries}
\renewcommand{\cftafterloftitle}{\hfill}

\setlength{\cftbeforefigskip}{10pt}
\cftsetrmarg{1.0in}

\begin{singlespace}
\listoffigures
\end{singlespace}
\clearpage

%%%%%%%%%%%%%%%%%%%%%%%%%%%%%%%%% LIST OF ABBREVIATINS %%%%%%%%%%%%%%%%%%%%%%%%%
\phantomsection
\addcontentsline{toc}{chapter}{LIST OF ABBREVIATIONS}

\begin{center}
{\normalsize \textbf{LIST OF ABBREVIATIONS}}
\end{center}

\newcommand{\Ab}[2]{\noindent  #1 \> #2 \\}
\newcommand{\Abi}[2]{\noindent #1 \hspace{1.5cm} \= #2 \\}

\begin{tabbing}
\Abi{LM}{Linear Model}\Abi{GLM}{Generalized Linear
Model}\Abi{TS}{Thompson Sampling}\Abi{RAR}{Response-adaptive
Randomization}
\end{tabbing}

\clearpage
%------------ END FRONTMATTER -------------------------------------------------%

%------------ BEGIN  MAIN TEXT ------------------------------------------------%

\pagenumbering{arabic}

\hypertarget{introduction}{%
\section{Introduction}\label{introduction}}

Test abstract

\abstract{Testy Mctest}

\hypertarget{getting-started}{%
\subsection{Getting started}\label{getting-started}}

\hypertarget{or-in-separate-files}{%
\subsection{Or in separate files}\label{or-in-separate-files}}

\hypertarget{literature-review}{%
\section{Literature Review}\label{literature-review}}

Or you can split chapters/sections/paragraphs or even words in separate
child documents.

\hypertarget{how-to-cite}{%
\section{How to cite}\label{how-to-cite}}

You can also cite other papers.

\hypertarget{rmarkdown-syntax}{%
\subsection{rmarkdown syntax}\label{rmarkdown-syntax}}

You cite articles using \texttt{rmarkdown} syntax (google rmarkdown for
helpful hints). For example, \citet{halmos1970howto} is a good text
read. Or writing about mathematics is challenging
\citep{halmos1970howto}.

%------------ END MAIN TEXT ---------------------------------------------------%
%%%%%%%%%%%%%%%%%%%%%%%%%%%%%%%%%%%%%%%%%%%%%%%%%%%%%%%%%%%%%%%%%%%%%%%%

%------------ BIBLIOGRAPHY ----------------------------------------------------%

\clearpage
\phantomsection

{\def\chapter*#1{} % suppress bibliograph header.
\begin{singlespace}
\addcontentsline{toc}{chapter}{BIBLIOGRAPHY}
\begin{center}
\normalsize \textbf{BIBLIOGRAPHY}
\vspace{17pt}
\end{center}

\bibliographystyle{apalike}
\bibliography{dissertation.bib}
\end{singlespace}
}



\end{document}
