\documentclass[12pt,,letterpaper,twoside]{report}
\usepackage[]{graphicx}
\usepackage[]{color}

%% maxwidth is the original width if it is less than linewidth
%% otherwise use linewidth (to make sure the graphics do not exceed the margin)
\makeatletter
\def\maxwidth{ %
  \ifdim\Gin@nat@width>\linewidth
    \linewidth
  \else
    \Gin@nat@width
  \fi
}
\makeatother

\definecolor{fgcolor}{rgb}{0.345, 0.345, 0.345}
\newcommand{\hlnum}[1]{\textcolor[rgb]{0.686,0.059,0.569}{#1}}%
\newcommand{\hlstr}[1]{\textcolor[rgb]{0.192,0.494,0.8}{#1}}%
\newcommand{\hlcom}[1]{\textcolor[rgb]{0.678,0.584,0.686}{\textit{#1}}}%
\newcommand{\hlopt}[1]{\textcolor[rgb]{0,0,0}{#1}}%
\newcommand{\hlstd}[1]{\textcolor[rgb]{0.345,0.345,0.345}{#1}}%
\newcommand{\hlkwa}[1]{\textcolor[rgb]{0.161,0.373,0.58}{\textbf{#1}}}%
\newcommand{\hlkwb}[1]{\textcolor[rgb]{0.69,0.353,0.396}{#1}}%
\newcommand{\hlkwc}[1]{\textcolor[rgb]{0.333,0.667,0.333}{#1}}%
\newcommand{\hlkwd}[1]{\textcolor[rgb]{0.737,0.353,0.396}{\textbf{#1}}}%
\let\hlipl\hlkwb

\usepackage{framed}
\makeatletter
\newenvironment{kframe}{%
 \def\at@end@of@kframe{}%
 \ifinner\ifhmode%
  \def\at@end@of@kframe{\end{minipage}}%
  \begin{minipage}{\columnwidth}%
 \fi\fi%
 \def\FrameCommand##1{\hskip\@totalleftmargin \hskip-\fboxsep
 \colorbox{shadecolor}{##1}\hskip-\fboxsep
     % There is no \\@totalrightmargin, so:
     \hskip-\linewidth \hskip-\@totalleftmargin \hskip\columnwidth}%
 \MakeFramed {\advance\hsize-\width
   \@totalleftmargin\z@ \linewidth\hsize
   \@setminipage}}%
 {\par\unskip\endMakeFramed%
 \at@end@of@kframe}
\makeatother

\definecolor{shadecolor}{rgb}{.97, .97, .97}
\definecolor{messagecolor}{rgb}{0, 0, 0}
\definecolor{warningcolor}{rgb}{1, 0, 1}
\definecolor{errorcolor}{rgb}{1, 0, 0}
\newenvironment{knitrout}{}{} % an empty environment to be redefined in TeX

\usepackage{alltt}

% Layout
\usepackage{geometry}
\usepackage{setspace}
\usepackage{titlesec}
\usepackage[subfigure]{tocloft}
\usepackage{mdframed}
\usepackage{titletoc}

% Citation style
\usepackage{natbib}
\usepackage{apalike}

% include citations inline
\usepackage{bibentry}
\nobibliography*

% Algorithms
\usepackage[ruled, vlined]{algorithm2e}


% Figures
\usepackage{subfigure}
\usepackage{epsfig}
\usepackage{booktabs}
\usepackage{multicol}
\usepackage{listings}

% Math
\usepackage{amsthm}
\usepackage{amsmath}
\usepackage{amssymb}

% Typography
\usepackage{times}
\usepackage{microtype}
\usepackage{textcomp}

% Macro support
\usepackage{xspace}

% PDF links
\usepackage[hidelinks]{hyperref} % backref=page

% Some packages to help with tables/figures
\usepackage[section]{placeins}
\usepackage{morefloats}
\usepackage{amssymb}
\usepackage{setspace}
\usepackage{rotating}
\usepackage{graphicx}
\usepackage{booktabs} 
\usepackage{stmaryrd}
\usepackage{times}
\usepackage{bm}
\usepackage{indentfirst}

% Use this command to start each chapter
% The format allows chapters in the table 
% of contents to meet grad school requirements
% as of 2017

\newcommand{\mychapter}[2]{
    \setcounter{chapter}{#1}
    \setcounter{section}{0}
    \chapter*{#2}
    \addcontentsline{toc}{chapter}{#2}
}


\usepackage{appendix}

\input{DoctoralThesisTexTemplate/template/layout}

\input{DoctoralThesisTexTemplate/template/macros}

\IfFileExists{upquote.sty}{\usepackage{upquote}}{}

% You can put custom definitions and other LaTeX header information here

% Operators
\DeclareMathOperator{\E}{{\rm I\kern-.3em E}}
\DeclareMathOperator{\I}{{\rm I}}
\DeclareMathOperator{\V}{{\mathcal{V} }}

\DeclareMathOperator{\BR}{BR}
\DeclareMathOperator{\R}{\rm R}
\DeclareMathOperator{\Reg}{\rm Reg}

\DeclareMathOperator*{\argmax}{argmax}
\DeclareMathOperator*{\argmin}{argmin}
\newcommand\norm[1]{\left\lVert#1\right\rVert}

% Useful shorthand
\newcommand{\mcA}{{\mathcal{A}}}
\newcommand{\mcE}{{\mathcal{E}}}
\newcommand{\mcF}{{\mathcal{F}}}
\newcommand{\mcH}{{\mathcal{H}}}
\newcommand{\mcV}{{\mathcal{V}}}
\newcommand{\mcX}{{\mathcal{X}}}

% High Probability Events 
\newcommand{\hpeDP}{{\mathcal{E}_{\mathrm{dp}}}}
\newcommand{\hpeMt}{{\mathcal{E}_{\text{MSE}_t}}}
\newcommand{\hpeW}{{\mathcal{E}_{\mathrm{W}}}}

\begin{document}

%--------   BEGIN FRONTMATTER -------------------------------------------------%
% front matter pages use 2in top margin
\newgeometry{left=1.25in,top=2in,right=1.25in,bottom=1in,nohead}
\pagenumbering{roman}

%%%%%%%%%%%%%%%%%%%%%%%%%%%%%%%%% TITLEPAGE %%%%%%%%%%%%%%%%%%%%%%%%%%%%%%%%%%%%
\begin{titlepage}
\begin{center}

% 1. The title of the thesis/dissertation, centered 2? below the top of the page

\vspace{2in}
\begin{singlespace}
Experimental Designs for Precision Medicine
\end{singlespace}


% 2. Your name, centered 1? below the title.
\vspace{61pt} % 1 in = 72pt, 11pt for the line with text
John H. Sperger
\end{center}

\vspace{50pt}
\begin{singlespace}
\begin{center}
\noindent 
A dissertation submitted to the faculty of the University of North Carolina at Chapel Hill in partial fulfillment of the requirements for the degree of Doctor
of Philosophy in the Department of Biostatistics in the Gillings School
of Global Public Health.
\end{center}
\end{singlespace}


%4. On the lower half of the page, centered, the words ?Chapel Hill?
%and one line below that, the year in which your committee approves
%the completed thesis/dissertation.
\vspace{50pt}
\begin{center}
\begin{singlespace} 
Chapel Hill\\
2022
\end{singlespace}
\end{center}

%5. On the right-hand side of the page, ?Approved by,? followed by lines for the
%signatures of the adviser and four (two for thesis) readers. List
 

\vfill
\begin{flushright}
\begin{minipage}[t]{1.5in} 
Approved by:\\
%To be approved by: \\
 
 Dr.~Michael R. Kosorok \\  Dr.~Eric Laber \\ 
 Dr.~Anastasia Ivanova \\  Dr.~Lisa M. Lavange \\  Dr.~Angela Smith \\ 

\end{minipage}
\end{flushright}

\end{titlepage}

%%%%%%%%%%%%%%%%%%%%%%%%%%%%%%%%% COPYRIGHT %%%%%%%%%%%%%%%%%%%%%%%%%%%%%%%%%%%%

%2. Copyright Page (optional)
\newgeometry{left=1.25in,top=8.33in,right=1.25in,bottom=1in,nohead}

%If you wish to copyright your thesis, you must include a copyright page with the following information single-spaced and centered on the bottom half of the page:
%? Year 
%Full Name (exactly as it appears on the title page) 
%ALL RIGHTS RESERVED
%This page should immediately follow the title page, and should bear the lower case Roman numeral: ii.

\begin{center}
\begin{singlespace}
\copyright 2022 \\
John H. Sperger \\
ALL RIGHTS RESERVED
\end{singlespace}
\end{center}

\clearpage
\newgeometry{left=1.25in,top=2in,right=1.25in,bottom=1in,nohead}

% Normal pages from here on out; TOC title takes care of 2in requirement.
\restoregeometry

%%%%%%%%%%%%%%%%%%%%%%%%%%%%%%%%% ABSTRACT %%%%%%%%%%%%%%%%%%%%%%%%%%%%%%%%%%%%
%The word Abstract should be centered below the top of the page. 
%Skip one line, then center your name followed by the title of the 
%thesis/dissertation. Use as many lines as necessary. Centered below the 
%title include the phrase, in parentheses, ?(Under the direction of  
%_________)? and include the name(s) of the dissertation advisor(s).
%Skip one line and begin the content of the abstract. It should be 
%double-spaced and conform to margin guidelines. An abstract should not 
%exceed 150 words for a thesis and 350 words for a dissertation. The 
%latter is a requirement of both the Graduate School and UMI's 
%Dissertation Abstracts International.
%Because your dissertation abstract will be published, please prepare and 
%proofread it carefully. Print all symbols and foreign words clearly and 
%accurately to avoid errors or delays. Make sure that the title given at 
%the top of the abstract has the same wording as the title shown on your 
%title page. Avoid mathematical formulas, diagrams, and other 
%illustrative materials, and only offer the briefest possible description 
%of your thesis/dissertation and a concise summary of its conclusions. Do 
%not include lengthy explanations and opinions.
%The abstract should bear the lower case Roman number ii (if you did not 
%include a copyright page) or iii (if you include a copyright page).

\begin{center}
\vspace*{52pt}
{\normalsize \textbf{ABSTRACT}}
\vspace{11pt}

\begin{singlespace}
 John H. Sperger : Experimental Designs for Precision Medicine \\
(Under the direction of   Dr.~Michael R. Kosorok and  Dr.~Eric Laber)
\end{singlespace}
\end{center}

Traditional clinical trial designs focus on demonstrating the efficacy
of a treatment compared to a placebo quantified in terms of the average
treatment effect (ATE). While these methods have been profoundly
successful, the guidance they provide is limited compared to the rich
complexity of information physicians consider when making their
treatment recommendations. In recent decades, two separate strands of
research have sought to build on these foundations while providing
evidence that can better inform clinical decision-making: 1) comparative
effectiveness studies that investigate the relative efficacy of multiple
treatments for the same condition in terms of the ATE, and 2) precision
medicine research that attempts to formalize the long-standing medical
practice of tailoring treatment based on an individual's unique
characteristics in a data-driven way. Equally allocating patients
between treatments is not an efficient way to investigate these new
research questions, and new experimental designs are needed to improve
the information gained from clinical trials. ¶ We first tackle the
question of how to design a trial when treatment effects vary depending
on a patient's characteristics and our goal is to estimate a dynamic
treatment regime (DTR) that will maximize the expected outcomes of the
general population. To this end, we propose a sequential design that
removes ineffective treatments at set intervals in the trial by
evaluating the change in the value function if that treatment weren't
available to assign in a DTR. This can be viewed as an extension of the
successive rejection algorithm from the response-adaptive randomization
(RAR) literature to the case where the expected response depends on
contextual patient information. We discuss a trial design problem that
motivated this design and conduct a simulation study to demonstrate the
effectiveness of this design compared to equal allocation. ¶ For further
projects, I propose to extend this design to the structured case where
one treatment can provide information about the expected response on
another treatment (e.g.~different doses of the same medication).
Eliminating ineffective treatments in the structured case will require
modifying the decision criteria to account for potential information
gain. I also propose a design for multi-level interventions that have an
intervention at the individual level that can be fully randomized and a
cluster-level intervention that is randomized using a stepped-wedge
design. This is motivated by the ongoing NC Works4Health trial to
evaluate the effectiveness of a multi-level intervention comprised of a
lifestyle coaching program at the individual level and supervisor
training at the employer level.

\clearpage

%%%%%%%%%%%%%%%%%%%%%%%%%%%%%%%%% DEDICATION %%%%%%%%%%%%%%%%%%%%%%%%%%%%%%%%%%%

%A dedication is an honorific statement from the author to a person or group to 
%whom the author commends the effort and product of the dissertation. Most 
%dedications are short statements of tribute beginning with ?To??. No heading is 
%required on the dedication page. The text of short dedications should be 
%centered between the left and right margins and 2? from the top of the page.
\begin{center}
\vspace*{52pt}

\singlespacing

In loving memory of Connie and Herb Sperger.

\end{center}

\pagebreak

%%%%%%%%%%%%%%%%%%%%%%%%%%%%%%%%% ACKNOWLEDGEMENTS %%%%%%%%%%%%%%%%%%%%%%%%%%%%%

%Acknowledgements are the author's statement of gratitude to and
%recognition of the people and institutions who helped the author's
%research and writing.
\begin{center}
\vspace*{52pt}
{\normalsize \textbf{ACKNOWLEDGEMENTS}}
\end{center}

To be acknowledged at a later date.

\clearpage

%%%%%%%%%%%%%%%%%%%%%%%%%%%%%%%%% TABLE OF CONTENTS %%%%%%%%%%%%%%%%%%%%%%%%%%%%

\renewcommand{\contentsname}{TABLE OF CONTENTS}
\renewcommand{\cfttoctitlefont}{\normalsize\bfseries}
\renewcommand{\cftaftertoctitle}{\hfill}
\renewcommand{\cftdotsep}{1.5}
\cftsetrmarg{1.0in}

\setlength{\cftbeforetoctitleskip}{61pt}
\setlength{\cftaftertoctitleskip}{28pt}



% format chapter entries like other entries
\renewcommand{\cftchapfont}{\normalfont}
\renewcommand{\cftchappagefont}{\normalfont}
\renewcommand{\cftchapleader}{\cftdotfill{\cftdotsep}}

\setlength{\cftbeforechapskip}{15pt}
\setlength{\cftbeforesecskip}{10pt}
\setlength{\cftbeforesubsecskip}{10pt}
\setlength{\cftbeforesubsubsecskip}{10pt}

\begin{singlespace}
\begin{center}
\tableofcontents
\end{center}
\end{singlespace}

\clearpage

%%%%%%%%%%%%%%%%%%%%%%%%%%%%%%%%% LIST OF TABLES %%%%%%%%%%%%%%%%%%%%%%%%%%%%%%%
\renewcommand{\listtablename}{LIST OF TABLES}
\phantomsection
\addcontentsline{toc}{chapter}{LIST OF TABLES}

\setlength{\cftbeforelottitleskip}{-11pt}
\setlength{\cftafterlottitleskip}{22pt}
\renewcommand{\cftlottitlefont}{\hfill\normalsize\bfseries}
\renewcommand{\cftafterlottitle}{\hfill}

\setlength{\cftbeforetabskip}{10pt}

\begin{singlespace}
\listoftables
\end{singlespace}

\clearpage

%%%%%%%%%%%%%%%%%%%%%%%%%%%%%%%%% LIST OF FIGURES %%%%%%%%%%%%%%%%%%%%%%%%%%%%%%
\renewcommand{\listfigurename}{LIST OF FIGURES}
\phantomsection
\addcontentsline{toc}{chapter}{LIST OF FIGURES}

\setlength{\cftbeforeloftitleskip}{-11pt} %11
\setlength{\cftafterloftitleskip}{22pt} %22
\renewcommand{\cftloftitlefont}{\hfill\normalsize\bfseries}
\renewcommand{\cftafterloftitle}{\hfill}

\setlength{\cftbeforefigskip}{10pt}
\cftsetrmarg{1.0in}

\begin{singlespace}
\listoffigures
\end{singlespace}
\clearpage

%%%%%%%%%%%%%%%%%%%%%%%%%%%%%%%%% LIST OF ABBREVIATINS %%%%%%%%%%%%%%%%%%%%%%%%%
\phantomsection
\addcontentsline{toc}{chapter}{LIST OF ABBREVIATIONS}

\begin{center}
{\normalsize \textbf{LIST OF ABBREVIATIONS}}
\end{center}

\newcommand{\Ab}[2]{\noindent  #1 \> #2 \\}
\newcommand{\Abi}[2]{\noindent #1 \hspace{1.5cm} \= #2 \\}

\begin{tabbing}
\Abi{CMAB}{Contextual multi-armed bandit}\Abi{LM}{Linear
Model}\Abi{GLM}{Generalized Linear Model}\Abi{MAB}{Multi-armed
Bandit}\Abi{SMART}{Sequential multiple-assignment randomized
trial}\Abi{TS}{Thompson Sampling}\Abi{RAR}{Response-adaptive
Randomization}
\end{tabbing}

\clearpage
%------------ END FRONTMATTER -------------------------------------------------%

%------------ BEGIN  MAIN TEXT ------------------------------------------------%

\pagenumbering{arabic}

\hypertarget{introduction}{%
\section{Introduction}\label{introduction}}

How can we ensure that patients receive the right medical interventions
at the right time? And beyond the individual, how can we determine where
to intervene to promote public health? These questions cut to the core
of medical and public health aspirations, but it is only in the past few
decades that methodological and technological advances have progressed
to the point where these questions can be investigated in a data-driven,
empirical way. These questions have been considered by both fields since
their inception, but this kind of tailoring has historically only been
possible in non-systematic and qualitative ways. While clinicians make
treatment decisions after eliciting details of a patient's disease
history and considering their characteristics and comorbidities, the
vast majority of evidence from randomized clinical trials (RCTs) are
very stylized circumstances where efficacy for an intervention was
compared to a placebo and extensive inclusion/exclusion criteria are
used to reduce heterogeneity as much as possible. New trial designs are
needed to bridge this gap.

Precision medicine, and precision health more broadly, endeavors to use
statistical methods to match ``the right patient to the right treatment
at the right time'' in an evidence-based way
\citep{kosorok2015adaptive, kosorok2019precision}. The majority of this
literature has focused on how to estimate optimal dynamic treatment
regimes including Q-learning, A-learning, Outcome Weighted Learning, and
Residual Weighted Learning and many others including extensions of these
methods. The major design innovation from this movement is the
Sequential multiple-assignment randomized trial (SMART), which was
developed to provide information that could be used for sequencing
treatments \citep{lei2012smart}. The randomization probabilities in a
SMART are typically static throughout the course of the trial, but the
randomization probabilities for later stages can depend on the
assignment and observed response from earlier stages. Trial designs for
the single-stage decision problem where the expected treatment response
is a function of patient covariates is relatively underexplored.
Interaction effects are necessarily more difficult to estimate than main
effects, and more efficient trial designs are needed to answer questions
about tailoring treatments to patients.

This dissertation proposes multiple randomization methods that can be
used to improve the estimation of dynamic treatment regimes (DTRs) and
evaluate multilevel interventions that would improve the efficiency of
trying to answer scientific questions about where and when to intervene
to maximally improve the health of individuals and consequently the
population. The next section reviews the existing literature on and
history of precision medicine, response-adaptive randomization methods
that change allocation probabilities based on accumulating evidence, and
the parallel literature on ``bandit'' problems that has evolved
primarily in the computer science and operations research literatures.

Chapter one extends the ``drop the loser'' rule to the general case
where the treatment effect can be a function of patient covariates
\citep{ivanova2000drop, ivanova2003play}; it can also be viewed as
extending the work on ``best arm identification'' for the multi-armed
bandit (MAB) problem to the contextual multi-armed bandit problem
(CMAB).

In chapter two, we extend this approach to the case where there is
structure between interventions and information from one treatment
assignment could provide information about the expected response on a
different treatment. This setting naturally arises in dose finding
studies where the response at one dosage can provide information about
the expected response at a different dosage. We consider two cases: one
where there is a safety constraint and one without a safety constraint.
In the case where there is no safety constraint, treatments that are
known to be suboptimal can still be assigned if they are more
informative about the optimal dose than other treatments. The structure
of the dose-response relationship determines whether this is the case;
in the simple case where there is a linear relationship between dose and
response the optimal design would only allocate patients to the highest
or lowest dose for example.

We investigate a trial design for multilevel interventions when the
individual component can be completely randomized but all participants
at the group level must receive the intervention. With these constraints
at the group level we propose a Stepped Wedge design where the time at
which a group receives the intervention is randomized
\citep{gambia1987gambia, hussey2007design}. We show the efficacy of this
method through simulation studies and provide a formula for sample size
calculation.

\hypertarget{literature-review}{%
\section{Literature Review}\label{literature-review}}

\hypertarget{terminology}{%
\section{Terminology}\label{terminology}}

Before diving in, a brief note about terminology. The problems
investigated in this dissertation have a rich history that spans
multiple research areas including statistics, operations research, and
computer science. With such a storied history spanning decades and
departments, it's perhaps no surprise that the terminology used in
different domains is not standardized, nor has it been standard across
time within each domain. The problem of how to sequentially choose an
action from a set of potential actions with unknown rewards where we
only observe the outcome for the action we chose is known as the
``multi-armed bandit'' (MAB) problem. The name is a reference to
one-armed bandits, a slang term for slot machines that first appeared
in, as far as I am aware, in \cite{bush1953stochastic}. While this is
the first paper the authors know of that uses the term, it seems likely
that the term was in circulation before this publication because the
authors didn't feel the need to define their terminology. Two decades
before this problem was named the MAB problem, William R. Thompson
proposed a heuristic solution for the two-arm case which adapts the
randomization probabilities based on the posterior probability that the
arm's mean reward is greatest \cite{thompson1933likelihood}. Related
work in this interim period simply referred to the problem under the
broad heading of ``sequential design''
\citep{wald1947sequential, robbins1952some}. When the response is a
function of covariates and the action, instead of simply the selected
action, the problem is most commonly called the contextual multi-armed
bandit (CMAB) problem \citep{woodroofe1979one, langford2007epoch}.

MAB and CMAB are the standard terminologies in computer science and
operations research, but recent statistical work does not have a common
term for these problem setups. The methods are often described by the
term response-adaptive randomization (RAR) to refer to the case where
the treatment assignment only depends on past actions and responses, and
covariate-adjusted response-adaptive (CARA) when covariate information
is included \citep{hu2006theory}. Depending on the author,
response-adaptive randomization may be used as an umbrella term for any
randomization method that uses past response information, while others
reserve the term for only those methods that use response but not
covariate information \citep{hu2006theory}. The even broader term
``adaptive randomization'\,' is also in use \cite{thall2015statistical}.
This is far from an exhaustive list of the terms authors have used to
describe these problems, but we hope this historical aside will serve as
a useful orientation for those who would like to read more on the
subject in different fields.

I will use the terms multi-armed bandit (MAB) and contextual multi-armed
bandit (CMAB) when referring to the respective problem formulations
because of the long history of the problem under these names. The terms
action, arm, intervention, and treatment will be used interchangeably
throughout the dissertation to refer to the treatment assignment.
Similarly context may be used in place of covariate information, and
reward in place of response when a larger response implies a better
outcome.

\subsubsection{Landau Notation}

Algorithms and randomization methods may be compared in terms of how
quickly the regret grows as a function of the number of arms and study
participants. To describe these bounds we'll use Landau notation or
``big o'\,'-notation to characterize the rate that functions grow. For
functions \(f, g\) that map the natural numbers to the non-negative
reals, we say that \(f(n)\) is \(O\left(g(n)\right)\) if and only if
\(\limsup_{n \to \infty}\frac{f(n)}{g(n)} < \infty\). While constants
may differ and smaller order terms may differ, the leading term in
\(f(n)\) can't grow at a faster rate than the leading term in \(g(n)\).
We say that \(f(n)\) is \(\Omega \left(g(n)\right)\) if and only if
\(\liminf_{n \to \infty}\frac{f(n)}{g(n)} > 0\).

\hypertarget{precision-medicine}{%
\section{Precision Medicine}\label{precision-medicine}}

Precision medicine, and more recently precision health out of
recognition for the broader applications of these methods beyond
clinical medicine, is a movement to formalize the long-standing practice
of physicians tailoring their treatments to the characteristics of their
patients and the disease presentation. The goal is to improve the health
of individuals and populations by taking heterogeneity into account in
an empirically-driven and statistically-sound way
\citep{kosorok2015adaptive, kosorok2019precision}. The apex of this
approach is decision support tools, and perhaps in the future, automated
decision making in appropriate contexts \citep{sperger2020future}.

The dynamic treatment regime (DTR) is perhaps the best known way to
formalize the map from patient characteristics to potential treatments
in Biostatistics, though it is also called an individualized treatment
rule or a policy. A DTR \(\pi\) is a function, or set of functions, that
maps patients to treatments based on their covariates
\(\pi: \mcX \to \mcA\). The data available to base our decisions on
comes is assumed to come in the form of a triple \((X, A, Y)\) where
\(X\) is a vector denoting the patient's covariate information, \(A\)
the action taken, and \(Y\) the response observed. Then the history at
time \(t\) \(\mcH_t\) is a filtration that encapsulates all the complete
data available
\(\mcH_t = \sigma\left((X_1, A_1, Y_1), \ldots, (X_t, A_t, Y_t)\right)\).
Letting \(Y^*\left(\pi(X)\right)\) denote the potential outcome if a
patient with covariates \(X\) were treated according to DTR \(\pi\), the
value of a DTR \(\mcV(\pi)\) is defined as the expected value of
assigning treatments according to the DTR:
\(mcV(\pi) = \E_X[Y^*(\pi(X))]\). The statistical goal of precision
medicine is to estimate an optimal DTR \(\pi^*\) that maximizes the
value function \(\mcV(pi^*) \geq \mcV(\pi) \,\forall \pi \in \Pi\). We
say an optimal DTR because there does not need to be a unique optimal
DTR, though many authors may impose this assumption for simplicity.

\hypertarget{estimation-methods}{%
\subsection{Estimation Methods}\label{estimation-methods}}

Estimation approaches for DTRs fall into two broad approaches: 1) the
``direct'' approach tries to optimize the estimated policy directly
without estimating the relationship between treatment and response, and
2) the ``indirect'' or regression-based approach which attempts to model
the and then creates a DTR by selecting the treatment that maximizes the
estimated expected response

\hypertarget{smarts}{%
\subsection{SMARTs}\label{smarts}}

The idea of adapting

Thompson Sampling

Play the winner

\hypertarget{optimal-design}{%
\section{Optimal Design}\label{optimal-design}}

The word ``optimal'\,' is frequently encountered in the discussion of
trial designs. The optimal design literature has a rich history going
back until at least the pioneering work of Kirstine Smith in her 1918
dissertation \citep{smith1918standard}. Before delving into their
results, it is important to emphasize that optimality is always
contingent on an objective and the assumptions made about the problem.
This is at odds with the non-technical usage of optimal meaning the
``best possible'' because, outside of extremely trivial special cases, a
design that is optimal for one standard objective will \emph{not} be
optimal for a different objective
\citep{silvey1980optimal, bubeck2009pure}.

Additionally, the results from this literature are not always directly
applicable to clinical research. It is often assumed that we have full
control to choose a point to sample, while in clinical trials we must
take patients as they are (subject to certain inclusion/exclusion
criteria). Similarly rewards may be deterministic rather than
stochastic. These optimality results can still provide a useful target;
one technique for adaptive designs is to find the optimal design and
then modify the randomization to target it
\citep{atkinson1982optimum, atkinson1992optimum}.

\hypertarget{common-kinds-of-optimality}{%
\subsection{Common Kinds of
Optimality}\label{common-kinds-of-optimality}}

To review the basic objectives for optimal design, consider the problem
of estimating a linear response function \(\E[Y] = X^T \beta\). Perhaps
the most common objective in optimal design is \(d\)-optimality which is
motivated by the goal of minimizing the volume for the confidence
ellipsoid around \(\beta\). The volume of the confidence ellipsoid for
\(\beta\) is proportional to \(\det{(XX^T)}^{-1/2}\), and so maximizing
\(\det{(XX^T)}\) makes this ellipsoid as small as possible.

\[\text{d-optimal criterion:=} \max_{X \in \mathcal{X}}  \text{det}\left(XX^T\right)\]
The \(d\)-optimality criterion is concerned with the overall volume of
the confidence ellipsoid, while we might only be interested in a
particular point or region. This desire motivates \(c\)- and
\(V\)-optimality which try respectivelyto minimize the confidence region
at a point and in a particular region. For a given \(c \in \mathcal{X}\)
of interest, the \(c\)-optimality criterion is defined as
\(\argmin_{X \in \mathcal{X}} c^T(XX^T)^{-1}c\). The \(V\)-optimality
criterion is defined in a similar fashion except we'll define a region
of interest \(C \subseteq \mathcal{X}\), and then
\(\argmin_{X \in \mathcal{X}} \int_{c \in C}c^T(XX^T)^{-1}c \, \mathrm{d}F(c)\).
\(c\)- and \(V\)- optimal designs have an intuitive appeal because the
whole covariate space is not usually equally interesting; an attractive
property of \(d\)-optimality that is not shared by either \(c\)- or
\(V\)-optimality is that it is invariant to reparameterizations.

\hypertarget{optimal-design-in-clinical-trials}{%
\subsection{Optimal Design in Clinical
Trials}\label{optimal-design-in-clinical-trials}}

Atkinson

Elving Set

Barycentric Spanners

Geometric design

Very dependent on the shape of the function

When the response is a linear function of the parameters the optimal
design puts the entirety of the observations on the edges of the design
space. When the response is a quadtratic function, the optimal design
splits half of the observations between the edges of the design space
and the other half is placed in the middle of the design space.
Christine Smith worked out the optimal design for polynomial functions
up to degree six. The degree must be known in advance. The optimal
design for a lower degree polynomial will not be optimal for higher
degree polynomials; indeed it may not be estimable at all.

Maximizing the expected number of successes falls under the general
umbrella of maximizing the cumulative expected reward
\(\R_n = \E[\sum_{t = 1}^n Y_t]\). This objective is commonly restated
as minimizing the cumulative regret, where regret is defined as the
difference in expectation between the action we chose and the action
chosen by an oracle with knowledge of the best treatment in expectation.
The expected number of successes and expected regret both have Bayesian
and Frequentist versions which differ in terms of whether the regret is
defined with respect to fixed parameter(s) (Frequentist) or a
distribution over the parameter space (Bayesian). Notice that maximizing
the cumulative expected rewards only considers the outcomes of those
within the trial. There are two commonly used contrasting objectives
that only consider the outcomes after the trial is complete: 1) the
expected reward for the first out-of-sample patient, known as the
``simple'' regret, and 2) the posterior probability assigned to the
hypothesis that an arm is optimal for the true optimal arm. A less
common, but no less important, objective is maximizing the power of a
statistical test (the specifics of the test will depend on the problem
at hand).

For all objectives, there is work on instance-dependent (or
problem-dependent) bounds that characterize optimal performance on a
specific bandit instance and instance-independent bounds that provide
more general performance guarantees that don't rely on parametric
assumptions. While instance-independent bounds don't rely on parametric
assumptions, they are still a family of results and the bounds for the
same algorithm and general problem may vary depending on what
assumptions are made such as boundedness or the tail probabilities
(e.g.~sub-Gaussian, sub-exponential). We can additionally distinguish
between asymptotic and finite-sample results.

Regret may be analyzed asymptotically as \(n\) goes to infinity or for a
finite time horizon. Asymptotically we wish to find a policy \(\pi\)
that for all bandits \(\nu\) satisfies
\(\lim_{n \to \infty} \frac{R_n(\pi, \nu)}{f(n)} = 0\) where \(f(n)\) is
as small as possible. In the finite-horizon case we will instead seek
guarantees that with high probability the regret will not grow faster
than a certain rate with the sample size. Asymptotic results do not
provide any guarantees about the finite-time performance of an
algorithm; you could take an algorithm and add a fixed period where we
throw out the data without looking at it without changing the asymptotic
performance. Still, favorable asymptotic results are encouraging that
the algorithm may have good finite-time performance.

We will now formally define the regret of a policy or bandit algorithm.
Let \(\nu = \{P_a: a \in \mathcal{A}\}\) be a collection of
distributions (a stochastic bandit), and let \(\mu_a(\nu)\) denote the
expected reward of playing arm \(a\),
\(\mu_a(\nu) = \int_{-\infty}^{\infty} x \, \mathrm{d}P_a(x)\). Define
\(\mu^*(\nu) = \max_{a \in \mathcal{A}} \mu_a(\nu)\) as the mean reward
of the optimal arm. Then the regret of policy \(\pi\) on bandit instance
\(\nu\) is given by
\[\Reg_n(\nu, \pi) = n\mu^*(\nu) - \E_{\pi} \left[\sum_{t = 1}^n Y_t \right]\]

This can be rewritten in terms of the gaps between the mean of each arm
and the optimal arm and the expected number of times that the arm will
be assigned provided there is a countable number of arms and a finite
time horizon. Define the suboptimality gap for arm \(a\) as
\(\Delta_a(\nu) = \mu^*(\nu) - \mu_a(\nu)\), and let the sample size for
arm \(a\) after \(n\) patients have entered the study be denoted by
\(N_a(n) = \sum_{t = 1}^n I(A_t = a)\). Then we can rewrite the regret

\[\Reg_n = \sum_{a \in \mathcal{A}}\Delta_a \E_{\pi}[N_a(n)]\]

\hypertarget{regret-and-power}{%
\subsection{Regret and Power}\label{regret-and-power}}

One might hope that by increasing the expected number of successes
within the trial we are also increasing the probability that we can
identify the best treatment after the trial. Sadly this is not the case.
Adaptive exploration methods that try to maximize the expected number of
successes in a trial do so at the expense of power in the two-arm case
\citep{bubeck2009pure, wathen2017simulation}. The power of a statistical
test is the probability of rejecting the null hypothesis using that test
for a fixed alternative parameter value. Power is contingent on the
statistical test in question. Consider an experiment that can be
represented using the \(K\)-armed MAB problem and the two null
hypotheses we might wish to test after running an experiment, hypothesis
A \(H_0: m_1 = \ldots = \mu_K\) and hypothesis B \(H_0: m_1 > m_2\). A
design that assigned all patients equally between Treatments 1 and 2
would increase the power to test Hypothesis B relative to a design that
split patients equally between Treatments 1, \ldots, K while decreasing
the power to test Hypothesis A. For realistic experiments, designs that
maximize the power to test all potential hypotheses related to the study
outcomes are not possible. The impact on power is more complex when
there are multiple treatments because there are multiple potential
hypotheses of interest including pair-wise comparisons and a global
hypothesis that all the arms' means are equal. The power to compare the
best and second-best treatment may increase because the sample size for
these arms could be higher than under equal allocation, but the power
for comparisons between arms that are sub-optimal will likely be worse
than under equal allocation. The conflict in what's being optimized
remains, and methods that explicitly try to maximize power when there
are multiple treatment arms will outperform TS at the expense of the
expected number of successes in the trial.

To help understand the intuition for why methods that optimize the
expected number of successes are suboptimal, consider Neyman's classic
design for allocating samples when the population can be divided into
mutually exclusive strata and variance within each strata is known; in
the case of a clinical trial without covariates the treatment arms form
the strata. Initially developed in the context of survey sampling, for
clinical trials, Neyman allocation \cite{neyman1934aspects} assigns
patients to treatments proportional to the treatment's variance over the
sum of all treatments' variances. It can be proven that this allocation
is optimal for minimizing the variance of the estimator for the
difference in treatment effect provided the resulting allocation is an
integer solution (rounding to reach an integer solution may not be
optimal). For a \(k\)-arm trial where the responses follow Gaussian
distributions with mean \(\mu_k\) and standard deviation \(\sigma_k\)
and where a Wald test will be used to test for a difference in treatment
effect, the optimal allocation ratio \(\rho\) of patients assigned to
treatment \(i\) is

\[\rho^*_{\mathrm{Neyman}}(a_i) = \frac{\sigma_i}{\sum_{k = 1}^{K} \sigma_k}\]

\hypertarget{results-for-mab-problem}{%
\section{Results for MAB Problem}\label{results-for-mab-problem}}

\[H = \sum_{k = 1}^{K}\]

\hypertarget{drop-the-loser-rule-for-contextual-bandits}{%
\section{Drop-the-loser Rule for Contextual
Bandits}\label{drop-the-loser-rule-for-contextual-bandits}}

\hypertarget{introduction-1}{%
\section{Introduction}\label{introduction-1}}

\hypertarget{motivating-trial}{%
\subsection{Motivating Trial}\label{motivating-trial}}

This work was motivated by a comparative effectiveness trial design for
a study to investigate four interventions for reducing the discomfort
from office cystoscopy. Cystoscopy is a procedure where an endoscope is
passed through the urethra to directly inspect the bladder, and this
procedure is typically done without general anesthesia. It is one of the
most common outpatient procedures with more than one million individuals
undergoing the procedure every year in the United
States\cite{dartmouth_atlas_data_2021}, and for patients with bladder
cancer must regularly (often every three months) undergo the procedure
for surveillance. Numerous interventions have been evaluated and have
some evidence of efficacy in reducing discomfort: intra-urethral
lidocaine with longer than standard dwell time (\(\geq 10\) minutes
versus immediately before the procedure, as is common), real-time
visualization where the patient watches their cystoscopy on a monitor,
increasing hydrostatic pressure via ``bag squeeze'\,' when the
cystoscope traverses the urethral sphincter, and listening to calming
music \textbf{add cites for methods}. The vast majority of cystoscopy
studies enrolled only men, and some that enrolled women have found
different effects for men and women. In addition to biological sex, age
and previous cystoscopy experience may impact treatment response. The
potential treatments included each intervention alone and all of the
treatments in pairwise combinations, for example combining increased
intra-urethral lidocaine dwell time with visualization. The goal of the
study was to create clinical treatment guidelines for assigning future
patients to treatments based on phenotypic characteristics in order to
reduce discomfort as measured on a 10-point visual analog scale. In
statistical terms, the goal was to estimate a dynamic treatment regime
that would best reduce the pain experienced by cystoscopy patients in
the general population.

Our motivating trial has a number of salient considerations that
inspired our design: 1) the interventions involve treatments already
used (to varying degrees)in clinical care, 2) there are a large number
of potential interventions with four solo interventions and six
combinations of two interventions, and 3) existing experimental evidence
and biologically-motivated theorizing suggest that there are multiple
subgroups with different responses to treatment. Because the
interventions are already used in clinical practice there is less
concern about safety constraints. The large number of treatments suggest
that removing less effective treatments could improve power at the end
of the study, but the presence of heterogenous treatment effects
preclude using methods that eliminate treatments based on the average
treatment effect (ATE) because a treatment with a lower ATE may be
optimal for a subgroup of individuals. In light of these considerations
we developed a randomization algorithm that can remove poorly performing
arms in a way that respects hetergenous treatment effects by evaluating
the arms in terms of their contribution to the value function of the
estimated optimal dynamic treatment regime (DTR).

\hypertarget{past-work}{%
\subsection{Past Work}\label{past-work}}

The existing literature on design for precision medicine is
predominantly focused on Sequential Multiple Assignment Randomized
Trials (SMARTs) \citep{lei2012smart, kosorok2015adaptive} which are
multi-stage trials. In a SMART the treatment received at a later stage
may be a function of an individual's response to an earlier treatment,
but the randomization probabilities are typically fixed over time. What
little work has been done on adaptive trial designs for precision
medicine has focused on maximizing the expected outcomes within the
trial population \citep{guo2017subgroup}. This is a reasonable goal in
certain contexts, but it is necessarily suboptimal when the goal is to
maximize the expected outcomes in a target population.

\hypertarget{problem-definition}{%
\subsection{Problem Definition}\label{problem-definition}}

We will now formalize the statement of the problem. At time
\(t = 1, \ldots [T]\) a new subject enters the study and we observe
their covariate information encoded in a vector
\(x_t \in \mathcal{X} \subseteq \mathbb{R}^d\) where
\(X_1, \ldots, X_2\) are independent and identically distributed random
variables with context distribution \(\mathcal{D}\). After this
covariate information is observed, a treatment (arm, action)
\(a_t \in \mathcal{A}\) is assigned according to an assignment policy
\(\pi: \mathcal{X} \times \mathcal{H}_{t-1} \to \mathcal{A}\) where
\(\mathcal{H}_{t-1} = \left\{(x_1, a_1, r_1), \ldots, (x_{t - 1}, a_{t - 1}, r_{t -1})\right\}\)
is the collected history of contexts, actions, and rewards through time
\(t-1\) with \(\mathcal{H}_0 = null\) . The outcome (reward)
\(r_t \in [0,1]\) is immediately observed. The conditional expected
reward is a fixed but unknown function of the covariate and treatment
information \(\E[r_t |x_t, a_t] = f(x_t, a_t)\).

\hypertarget{objective}{%
\subsubsection{Objective}\label{objective}}

Our goal will be to minimize the expected regret in the target
population based on the estimated optimal policy \(\widehat{\pi}_T\) at
the conclusion of the study. This does not take into account any of the
outcomes in the study except insofar as they influence the estimation of
the optimal policy at the end of the trial.

\[\mathrm{Reg}_{\text{pop}} = \E_{X, \pi}[f^*(\pi^*(x), x) - f^*(\widehat{\pi}_T(x), x)]\]

\hypertarget{assumptions}{%
\subsubsection{Assumptions}\label{assumptions}}

\emph{Assumption - realizibility}

There exists \(f^* \in \mathcal{F}\) s.t.
\(\E[Y | x_t, a_t] = f^*(x_t, a_t)\)

\emph{Assumption - rewards}

\(Y_t = f^*(a_t, \pi_t) + \epsilon_t\)

Where \(\epsilon_t\) is a fixed but unknown distribution. Because
\(Y_t\) is bounded this implies that \(\epsilon_t\) has a sub-Gaussian
distribution.

\emph{Assumption - Function Class}

\(|\mathcal{F}|\) is bounded

\(\mathrm{VC}(\mathcal{F})\) is bounded where \(\mathrm{VC}\) is
Vapnik-Chernis dimension

\emph{Assumption - Oracle Access - Foster'20}

We assume that we have access to a solver

\(\widehat{f}_m = \argmin_{f \in \mathcal{F}} \sum_{t = \tau_{m-1}}^{\tau_m - 1} \left(f(x_t, a_t) - r(x_t, a_t) \right)^2\)

\hypertarget{method}{%
\subsection{Method}\label{method}}

The logic of our method is to increase the sample size for better
treatments at the expense of inferior treatments by sequentially
eliminating treatments that are not optimal for any subset of patients.
It proceeds by splitting the available samples \(T\) into \(M\) epochs
where \(M = ceiling(\log_2{T})\). At the end of an epoch the set of
feasible treatments for the next epoch, \(\mathcal{A}_{m + 1}\), is
formed by constructing confidence intervals around the estimated value
functions for the estimated optimal policies if treatment \(a\) were not
available
\(\pi_{\not a} = \pi: \mathcal{H} \times \mathcal{X} \to \mathcal{A} \, \backslash \, a\).
If a treatment is \emph{not} part of the optimal DTR then the value
function doesn't change when it is removed, that is
\(\mathcal{V}(\pi_{\not a}) = \mathcal{V}(\pi)\). For mathematical
simplicity, we will only use the observations in the most recent epoch
to determine the set of feasible treatments for the next epoch. Because
the assignments within an epoch are random with equal probability the
parameter estiamtes will be unbiased. Data from past epochs could be
incorporated using an AIPW estimator in practice.

\(\mathcal{V}(\pi) = \E_X[Y^*(\pi(X))]\)

\(\mathcal{V}(\pi_{\not a})\)

\hypertarget{meta-algorithm}{%
\subsubsection{Meta-algorithm}\label{meta-algorithm}}

\begin{algorithm}[H]
\SetAlgoVlined
\KwIn{Sample size $N$, Treatment arms $[K]$}
\SetKwFunction{FindDominatedArms}{FindDominatedArms}
\SetKwFunction{WithinEpochRandomization}{WithinEpochRandomization}

Set the number of epochs $L = \lceil \log_2{K} \rceil$

Define epoch boundaries $\tau_m = 2^m$

Set epoch sample sizes to $t_m = \tau_m - \tau_{m - 1}$

Set active arm set $\mathcal{A}_1 = [K]$

 \For{$l = 1, \ldots, L$}{
  Assign patients to $a \in \mathcal{A}_l$ using \WithinEpochRandomization 
  
Update active arm set $\mathcal{A}_{l + 1} = \mathcal{A}_l \, \backslash$ \FindDominatedArms{$\mathcal{A}_l, \, \mathcal{H}_t$}
  
 }
 \caption{Contextual Drop the Loser Meta-algorithm}
\end{algorithm}

\hypertarget{high-probability-events}{%
\subsection{High Probability Events}\label{high-probability-events}}

Taken from Foster 20 (add citation)

\hypertarget{hpedp}{%
\subsection{\texorpdfstring{\(\hpeDP\)}{\textbackslash hpeDP}}\label{hpedp}}

Set \(\mu_m = \frac{64 \log{4M/\delta}}{n_{m-1}}\) for all
\(m \in [M]\). Let \(\delta\) be a fixed in the interval \((0, 1]\), and
then define the event \(\hpeDP\) as the event that for all \(m \in [M]\)

\[\hpeDP := \{2/3 \varrho \leq \widehat{\varrho} \leq 4/3 \varrho\}\]

\(\hpeDP\) holds with probability at least \(1 - \delta/2\)

\hypertarget{hpew}{%
\subsection{\texorpdfstring{\(\hpeW\)}{\textbackslash hpeW}}\label{hpew}}

Let \(\delta\) be a fixed in the interval \((0, 1]\), and for all
\(m \in [M]\) let \(\hpeW^{(m)}\) denote the event that

\[\hpeW^{(m)} = \begin{cases} \widehat{w}_m \in [2/3 w_m, 4/3w_m] & \text{if } w_m \geq \frac{64}{n_{m-1}} \log{4M/\delta} \\ \widehat{w}_m < \frac{65}{n_{m-1}} \log{4M/\delta} & \text{if } w_m > \frac{64}{n_{m-1}} \log{4M/\delta}
\end{cases}\]

Define the event \(\hpeW = \cap_{m = 1}^{M} \hpeW^{(m)}\)

Then \(hpeW\) holds with probability at least \(1 - \delta/2\)

Let \(D_{\delta} = 16 \log{\frac{2 | \mcF | T^2 }{\delta}}\)

Define
\(M_t(f) = \left(f(x_t, a_t) - r_t(a_t) \right)^2 - \left(f^*(x_t, a_t) - r_t(a_t) \right)^2\)

Let \(\hpeMt\) denote the event that

\[\sum_{t = \tau}^{\tau'} \E[M_t(f) | \mcH_{t - 1} ] \leq C_{\delta} + 2 \sum_{t = \tau}^{\tau'}M_(f)\]

For all \(f \in \mcF\), and \(\tau, \tau' \in [T]\) \(\hpeMt\) holds
with probability at least \(1-\delta / 2\)

\hypertarget{proposal-two}{%
\section{Proposal Two}\label{proposal-two}}

Determining dosages that are both efficacious and safe is a key step in
drug development. This dosage is often dependent on patients'
characteristics such as age, weight, and disease-specific biomarkers.
Often these differences are dealt with qualitatively or ignored in
early-phase clinical trials. The question of estimating an optimal
personal dosing role has received some attention in the past with the
growing interest in precision medicine. This literature focused on
estimating personalized dose responses after data has been collected
\citep{chen2016personalized, zhu2020kernel, zhou2021parsimonious}. In
this project, we will instead investigate how to design a clinical trial
when the goal is to estimate an optimal dosing rule and we're able to
adapt the randomization probabilities over time.

\subsection{Previous work}

Dose-finding trials have a rich tradition of adaptivity from the
(deservedly) statistically maligned 3+3 design to more statistically
sound methods including Bayesian Continual Reassessment Method
\cite{o1990continual, goodman1995some}. More recently,
\cite{guo2017bayesian} propose an adaptive design when there are
biomarkers that affect dose response utilizing Thompson Sampling.
However, their objective is to maximize the expected number of successes
within the trial rather than most accurately estimate the rule so that
when it's applied to the general population it minimizes the expected
regret.

In the bandit literature, this type of problem is an example of a
structured bandit problem where playing one arm provides some
information about the expected reward of playing a different arm. The
linear bandit problem is the simplest structured bandit problem, and in
it, the expected response is given by \(a^T \theta\) where the parameter
vector \(\theta\) is shared across all actions, we have full control
over which \(a \in \mathcal{A} = \{a_1, \ldots, a_k\}\) we select, and
the response does not include any covariate information
\cite{xu2018fully}. The optimism under uncertainty principle that's used
in unstructured bandit problems like the MAB and CMAB problems are
suboptimal in a structured bandit problem \citep{lattimore2017end}.

\subsection{Problem Definition}

We consider a trial where the sample size \(T\) is fixed, there is a
single treatment with \(K\) possible dosages, and the researcher is able
to adapt the randomization probability for the \(t\)-th patient based on
the accumulated data of the trial up to that point. Each treatment \(k\)
is associated with an unknown parameter vector
\(\beta_k \in \mathbb{R}^d\). For any integer \(k\), let \([k]\) denote
the set \(\{1, \ldots, k \}\). At each time \(t = 1, \ldots, T\), we
observe a new individual with baseline covariates
\(x_t \in \mathbb{R}^d\). The experimenter then uses a randomization
policy \(\pi(x_t)\) to assign this individual to a treatment
\(a_t \in [K]\) and observes a stochastic linear outcome
\[Y_{t} = X_t^T \beta_k + A_t^T\theta + \epsilon_{t}\]

\subsection{Proposed Method}

Propose to extend the successive rejection algorithm to the structured
case by modifying the rejection criteria to include the potential
information gain that an arm provides.

\hypertarget{proposal-three}{%
\section{Proposal Three}\label{proposal-three}}

\hypertarget{appendix}{%
\section{Appendix}\label{appendix}}

\section{Notation}

For a natural number \(n\), let \([n]\) denote the set
\(\{1, \ldots, n\}\)

\hypertarget{optimal-design-1}{%
\subsection{Optimal Design}\label{optimal-design-1}}

\hypertarget{problem-definition-1}{%
\subsection{Problem Definition}\label{problem-definition-1}}

\begin{itemize}
    
\item $t \in [T]$ indexes the observations

\item $k \in [K]$ indexes arms 

\item $X_t \in  \mathcal{X} \subseteq \mathbb{R}^d$ denotes the context (covariate) vector at time $t$

We'll assume that $\mathcal{X}$ has been rescaled so that $\mathcal{X} \subseteq [0, 1]^d$ for simplicity

\item $A_t \in [K] \equiv \mathcal{A}$ denotes the arm (action, treatment) chosen at time $t$
     

 \item $\beta_k$ is the parameter vector associated with arm $k$

\item $Y_t \in \mathbb{R}$ denotes the outcome at time $t$

\item $\mathcal{H}_t = \sigma(X_1, A_1, Y_1, \ldots, A_{t - 1}, Y_{t -1}, X_t)$ the collected history available at prior to making the assignment for patient $t$
\end{itemize}

\hypertarget{algorithm-related}{%
\subsection{Algorithm-related}\label{algorithm-related}}

\hypertarget{tuning-parameters}{%
\paragraph{Tuning parameters}\label{tuning-parameters}}

\begin{itemize}
\item $\delta$ failure probability
\item $\beta_m$ Confidence radius
\item $\mu_m$ smoothing parameter
\item $\lambda_m$ instance dependent scale factor
\item $\gamma_m$ Learning rate. 

$$\gamma_m = \lambda_m \sqrt{\frac{| \mcA| n_{m-1}}{\log{2 | \mcF | T^2 / \delta}}}$$
\end{itemize}

%------------ END MAIN TEXT ---------------------------------------------------%
%%%%%%%%%%%%%%%%%%%%%%%%%%%%%%%%%%%%%%%%%%%%%%%%%%%%%%%%%%%%%%%%%%%%%%%%

%------------ BIBLIOGRAPHY ----------------------------------------------------%

\clearpage
\phantomsection

{\def\chapter*#1{} % suppress bibliograph header.
\begin{singlespace}
\addcontentsline{toc}{chapter}{BIBLIOGRAPHY}
\begin{center}
\normalsize \textbf{BIBLIOGRAPHY}
\vspace{17pt}
\end{center}

\bibliographystyle{apalike}
\bibliography{dissertation.bib}
\end{singlespace}
}



\end{document}
